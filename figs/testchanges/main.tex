%\documentclass[a4paper]{book}
\documentclass{ctexart}

% 可用颜色区分各作者、审阅者
\usepackage[dvipsnames]{xcolor}
\usepackage{changes}

% 设定作者颜色、id
\def\zhang{\protect\includegraphics[width=1.5cm]{xiaogege}}
\def\profwang{\protect\includegraphics[width=1.5cm]{hlz}}

\definechangesauthor[name=张生, color=BrickRed]{Zh}
\definechangesauthor[name=王教授, color=RoyalBlue]{profW}

\begin{document}

\highlight[id=Zh, comment=不错的学校!]{西北农林科技大学}地处中华农耕文明发祥地、国家级农业高新技术产业示范
区---陕西杨凌,教育部直属、国家“985工程”和“211工程”重点建设高校,首
批入选国家“世界一流大学和一流学科”建设高校。现任党委书记李兴旺、校长
吴普特。

学校前身是创建于1934年的国立西北农林专科学校\added[id=profW, comment=
第1任校长是谁?]{,第1任校长是于右任}。1999年9月,经国务院批准,
由原西北农业大学、西北林学院、中国科学院水利部水土保持研究所、水利部西
北水利科学研究所、陕西省农业科学院、陕西省林业科学院、陕西省中国科学院
西北植物研究所等7所科教单位合并组建为西北农林科技大学。

% 建校80余年来,学校一代代师生秉承“经国本,解民生,尚科学”的办学理念
% 和“诚朴勇毅”的校训,心怀社稷,情系苍生,承远古农神后稷之志,行当
% 代“教民稼穑”之为,坚持走产学研紧密结合的办学道路,为推动我国农业现代
% 化建设和农业科教事业发展做出了突出贡献。

学校在新中国成立前已是一所在国内外具有重要影响的知名大学;上世纪五六十
年代,学校事业经历了一个快速的发展阶段,取得了辉煌业绩。合校以来,学校
不断突出产学研紧密结合的办学特色,积极推进和深化科教体制改革,各项事业
均实现了历史性跨越式发展\deleted[id=profW, comment=表达清楚了!]{,进入了新的发展阶段}。

学校是全国农林水学科最为齐备的高等农业院校,设有26个学院(系、所、部)
和研究生院,共有13个博士后流动站,16个博士学位授权一级学科,28个硕士学
位授权一级学科,67个本科专业。现有7个国家重点学科和2个国家重点(培育)
学科;农业科学居US.NEWS学科排名全球第18位;农业科学学科领域进入ESI全球
学科排名前1‰之列,农业科学、植物学与动物学、工程学、环境科学与生态学、
化学、生物学与生物化学、药理学与毒理学等7个学科领域进入ESI全球学科排名
前1\%之列。建有2个国家重点实验室,1个国家工程实验室,3个国家工程技术研
究中心,3个国家野外科学观测研究站,62个省部重点实验室及工程技术研究中
心。

学校现有教职工4462人,其中专任教师\replaced[id=Zh, comment=请认真查
对!]{2183人}{2813人},正高级专业技术人员623人,副高
级专业技术人员1321人。 有中国科学院院士1人,中国工程院院士2人,双聘院
士11人;国家“人才项目专家”入选者7人,青年人才项目专家入选者13人;“长
江学者”特聘教授6人,青年长江学者3人;“万人计划”科技创新领军人才6人,
青年拔尖人才3人;国家杰出青年科学基金获得者8人,优秀青年基金获得者8人;
国家百千万人才工程入选者12人,新世纪优秀人才支持计划入选者64人;陕西
省“百人计划”入选者43人,学校“特聘教授”12人,国家教学名师2人
\comment[id=Zh]{强!}。

\end{document}
%%% Local Variables:
%%% mode: latex
%%% TeX-master: t
%%% End:
